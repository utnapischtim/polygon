\subsection{Space Partitioning}
\subsubsection{Description of the Algorithm}
The main idea is a combination of the divide and conquer principle and
the convex hull. The algorithm divides the point cloud by a line and
recursively process those sets. Finally the points of the sets were
pushed to the polygon. The fact that every set has a convex hull that
has only one point in common with adjacent sets guarantees a simple
polygon.
\\[12pt]
Following a description of the algorithm in detail:

\begin{enumerate}
  \item select two points
  \item divide the point cloud by a line through those points \fref{sp:line}
  \item recursively process those two sets of points \fref{sp:line2} to \fref{sp:line5}
  \begin{enumerate}
    \item exit point set of points == 2 || == 3 push points on the final list
    \item calculate a random point
    \item calculate a random line with this point
    \item divide the point cloud by that line
  \end{enumerate}
\end{enumerate}

\subsubsection{Implementation description}
\begin{enumerate}
  \item a random function selects two points ($s_l$ and $s_f$). Point $s_l$ is the
    first point in rotation direction and $s_f$ the second point.
  \item to divide the point cloud in two independent sets we create a
    line through the two points. The \textit{Line\_2} class from the cgal
    framework has a \textit{has\_on\_positive} and \textit{has\_on\_negative} function.
    With these it is possible to distinguish between a point of one or
    the other side of the line.
  \item with the created sets we call recursivelly the function
    \textit{recursiveDivide} and give the function the created set $S$, point $s_f$,
    point $s_l$ and the polygon $C$. One call has as first element $s_f$ the other has $s_l$
    as first element of the two points. This is necessary to close the
    polygon.
  \begin{enumerate}
    \item there are two exit points of the \textit{recursiveDivide} function.
      The first is with set \textit{S.size == 2} and pushs the point $s_l$ to the
      polygon $C$. The second is with set \textit{S.size == 3}. Here we have to
      search the third point $s$, which is not $s_l$ or $s_f$ and then we
      could push $s$ and $s_l$ to the polygon $C$.
    \item if the exit points does not occure we calculate a random
      point. The random point should be selected randomly. The
      condition is that it is not a point equal to and not collinear
      with $s_f$ and $s_l$. The next step is to calculate a random line
      through the point s and a point of the line through $s_f$ and
      $s_l$. Essentially is that $s_f$ lies on the positive side of the
      calculated random line.
    \item the division of the set is equal as in the parent function
  \end{enumerate}
\end{enumerate}

\subsubsection{Complexity}
$\bigO(n^2)$ if it is done in a good way then it should be $\bigO(nlogn)$

\subsubsection{Parameters}
\begin{description}
  \item [--nodes] how many nodes the polygon has to have. [default: 100]
  \item [--sampling-grid] the area within the polygon could grow. [default: 1500x800]
\end{description}

\subsubsection{Examples}

\begin{figure}[ht]
  \centering

  \begin{minipage}[t]{0.4\textwidth}
    \begin{tikzpicture}[yscale=0.05,xscale=0.05]
      \setcounter{i}{1}

      \draw[->] (0,0) -- (80,0) node[below] {$x$};
      \draw[->] (0,0) -- (0,70) node[left] {$y$};

      \foreach \p in {(5,40),(10,20),(20,41),(21,10),(33,15),(39,22),(48,9),(55,17),(68,19)} {
        \node[point] (\arabic{i}) at \p {};
        \stepcounter{i}
      }
    \end{tikzpicture}
    \caption{Base random point cloud}
    \label{fig:sp:base}
  \end{minipage}\hfill
  \begin{minipage}[t]{0.4\textwidth}
    \begin{tikzpicture}[yscale=0.05,xscale=0.05]
      \setcounter{i}{1}

      \draw[->] (0,0) -- (80,0) node[below] {$x$};
      \draw[->] (0,0) -- (0,70) node[left] {$y$};

      \foreach \p in {(5,40),(10,20),(20,41),(21,10)} {
        \node[point, fill=red] (\arabic{i}) at \p {};
        \stepcounter{i}
      }

      \foreach \p in {(33,15),(39,22)} {
        \node[point, fill=violet] (\arabic{i}) at \p {};
        \stepcounter{i}
      }

      \foreach \p in {(48,9),(55,17),(68,19)} {
        \node[point, fill=blue] (\arabic{i}) at \p {};
        \stepcounter{i}
      }

      \path (5) -- (6) coordinate[pos=-3.5](dd) coordinate[pos=10](ff);
      \draw[green, dashed] (dd) -- (5);
      \draw[green] (5) node[right=1pt, black] {$s_f$} -- (6) node[left=1pt, black] {$s_l$};
      \draw[green, dashed] (6) -- (ff);
    \end{tikzpicture}
    \caption{First two points with the line and the two sets. Points on the line are in both sets.}
    \label{fig:sp:line}
  \end{minipage}
\end{figure}

\begin{figure}[ht]
  \centering

  \begin{minipage}[t]{0.4\textwidth}
    \begin{tikzpicture}[yscale=0.05,xscale=0.05]
      \setcounter{i}{1}

      \draw[->] (0,0) -- (80,0) node[below] {$x$};
      \draw[->] (0,0) -- (0,70) node[left] {$y$};

      \foreach \p in {(5,40),(10,20),(20,41),(21,10),(33,15),(39,22),(48,9),(55,17),(68,19)} {
        \node[point] (\arabic{i}) at \p {};
        \stepcounter{i}
      }

      \draw[green] (5) node[right=1pt, black] {$s_f$} -- node[placeholder](a){} (6) node[right=1pt, black] {$s_l$};
      \draw[red] (a) -- (1) node[right=1pt, black] {$s$};
    \end{tikzpicture}
    \caption{First split of left set}
    \label{fig:sp:line2}
  \end{minipage}\hfill
  \begin{minipage}[t]{0.4\textwidth}
    \begin{tikzpicture}[yscale=0.05,xscale=0.05]
      \setcounter{i}{1}

      \draw[->] (0,0) -- (80,0) node[below] {$x$};
      \draw[->] (0,0) -- (0,70) node[left] {$y$};

      \foreach \p in {(5,40),(10,20),(20,41),(21,10),(33,15),(39,22),(48,9),(55,17),(68,19)} {
        \node[point] (\arabic{i}) at \p {};
        \stepcounter{i}
      }

      \draw (5) -- node[placeholder](a){} (6);
      \draw (a) -- node[placeholder](b){} (1);
      \draw (2) -- (b);
    \end{tikzpicture}
    \caption{Second split of left set}
    \label{fig:sp:line3}
  \end{minipage}
\end{figure}

\begin{figure}[ht]
  \centering

  \begin{minipage}[t]{0.4\textwidth}
    \begin{tikzpicture}[yscale=0.05,xscale=0.05]
      \setcounter{i}{1}

      \draw[->] (0,0) -- (80,0) node[below] {$x$};
      \draw[->] (0,0) -- (0,70) node[left] {$y$};

      \foreach \p in {(5,40),(10,20),(20,41),(21,10),(33,15),(39,22),(48,9),(55,17),(68,19)} {
        \node[point] (\arabic{i}) at \p {};
        \stepcounter{i}
      }

      \draw (5) -- node[placeholder](a){} (6);
      \draw (a) -- node[placeholder](b){} (1);
      \draw (2) -- (b);
      \draw (8) -- (a);
    \end{tikzpicture}
    \caption{First split of right set}
    \label{fig:sp:line4}
  \end{minipage}\hfill
  \begin{minipage}[t]{0.4\textwidth}
    \begin{tikzpicture}[yscale=0.05,xscale=0.05]
      \setcounter{i}{1}

      \draw[->] (0,0) -- (80,0) node[below] {$x$};
      \draw[->] (0,0) -- (0,70) node[left] {$y$};

      \foreach \p in {(5,40),(10,20),(20,41),(21,10),(33,15),(39,22),(48,9),(55,17),(68,19)} {
        \node[point] (\arabic{i}) at \p {};
        \stepcounter{i}
      }

      \draw (5) -- (4) -- (2) -- (1) -- (3) -- (6) -- (9) -- (8) -- (7) -- (5);
    \end{tikzpicture}
    \caption{Final polygon}
    \label{fig:sp:line5}
  \end{minipage}
\end{figure}

\FloatBarrier
