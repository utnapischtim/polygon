\subsection{Fix Local Orientation}
\subsubsection{Description of the Algorithm}
This algorithm has two steps. The first step is the creation of a regular
polygon. The second step is to include reflex nodes into the segments.
\\[12pt]
Following the description of the algorithm in detail:

\begin{enumerate}
  \item calculate the settings
  \item create the regular polygon
  \item create the reflex nodes
\end{enumerate}

\subsubsection{Implementation description}
A implementation detail in advance. It is helpfull that the regular
polygon algorithm creates a polygon where the first segment lies on a
vertical line, because only than the middle point could be calculated
in a simple way. It is not mandatory but it makes this algorithm easier to
implement.

\begin{enumerate}
  \item calculate the settings with the parameter given by the user or
    the default values. It is important to pay attention that the node count for
    the regular polygon is the node count set by the parameter \textit{--node}
    minus the reflex node count. This part creates also a random list how many
    reflex nodes has to be created in how many segments.
  \item the implementation of the regular polygon algorithm creates the regular
    polygon.
  \item the idea is to create the reflex nodes on the semi circle between the
    segment endpoints. This ensures not any intersection between any segment.
    Following a detailed description of this algorithm:
    \begin{enumerate}
      \item calculate the point in the middle of the first segment and
        the distance from the center to it.
      \item iterate over the segments
      \item wherever reflex nodes have to be added the vector to the
        middle point rotates about the angle from the first segment
        to the current segment to get the position of the middle point
        of the current segment.
      \item calculate the vector \textit{l} from the center of the current
        segment to the target point of the current segment.
      \item with the reflex node counts for this segment and the angle for every
        segment of the regular polygon it could be calculated the angle with
        which the vector \textit{l} could be rotated.
      \item then every reflex node is only a further rotation of the
        vector l.
    \end{enumerate}
\end{enumerate}

\subsubsection{Complexity}
$n$ stands for the node count and $k$ for the maximum possible reflex count per segment.
\begin{enumerate}
  \item calculate the settings $\bigO(1)$
  \item create the regular polygon $\bigO(n)$
  \item create the reflex nodes $\bigO(nk)$
\end{enumerate}
This leads to the sum of $\bigO(1) + \bigO(n) + \bigO(nk) \Rightarrow
max(\bigO(1) + \bigO(n) + \bigO(nk)) \Rightarrow \bigO(nk) \Rightarrow
\bigO(n)$. The value of $k$ and $n$ sums up to the number of nodes given by the
commandline argument. Therefore if $k << n$ then it is a constant factor which
apply every iteration. If $n << k$ then it is the same as before but in opposite
direction which leads to the same result. The last case is, if $k < n, k == n, k
> n$. This means, that if all reflex nodes where set to one segment than all
other segments have nothing to do and if all segments have one reflex node it is
again a constant factor.

\subsubsection{Parameters}
\begin{description}
  \item [--nodes] how many nodes the polygon has to have. [default: 100]
  \item [--sampling-grid] the area within the polygon could grow. [default: 1500x800]
  \item [--reflex-points] set the number of reflex points that should be at least. [default: 0]
  \item [--reflex-chain-max] the maximal length of a chain. the default value says that there is no limit. [default: -1]
  \item [--segment-length] set the segment length. [default: 0]
  \item [--radius] the radius for regular polygon. [default: 60]
\end{description}


\subsubsection{Examples}

\begin{figure}[ht]
  \centering

  \begin{minipage}[t]{0.4\textwidth}
    \begin{tikzpicture}[yscale=0.05,xscale=0.05]

      \setcounter{i}{1}

      \draw[->] (0,0) -- (80,0) node[below] {$x$};
      \draw[->] (0,0) -- (0,70) node[left] {$y$};

      \foreach \p in {(62,26),(51,11),(34,6),(16,11),(5,26),(5,45),(16,60),(34,66),(51,60),(62,45)} {
        \node[point] (\arabic{i}) at \p {};
        \stepcounter{i}
      }

      \draw (1) -- (2) -- (3) -- (4) -- (5) -- (6) -- (7) -- (8) -- (9) -- (10) -- (1);
    \end{tikzpicture}
    \caption{Polygon with convex nodes generated by the regular
      polygon algorithm}
    \label{fig:flo:base}
  \end{minipage}\hfill
  \begin{minipage}[t]{0.4\textwidth}
    \begin{tikzpicture}[yscale=0.05,xscale=0.05]

      \setcounter{i}{1}

      \draw[->] (0,0) -- (80,0) node[below] {$x$};
      \draw[->] (0,0) -- (0,70) node[left] {$y$};

      \foreach \p in {(59,26),(39,10),(14,16),(2,39),(14,63),(39,69),(59,52)} {
        \node[point,color=blue] (\arabic{i}) at \p {};
        \stepcounter{i}
      }

      \foreach \p in {(29,22),(28,57),(43,54)} {
        \node[point,color=red] (\arabic{i}) at \p {};
        \stepcounter{i}
      }

      \draw (1) -- (2) -- (8) -- (3) -- (4) -- (5) -- (9) -- (6) -- (10) -- (7) -- (1);
    \end{tikzpicture}
    \caption{Regular polygon filled up with reflex nodes[Reflex nodes
    red, convex nodes blue]}
    \label{fig:flo:line}
  \end{minipage}
\end{figure}

\FloatBarrier
