\subsection{Regular Polygon}
\subsubsection{Description of the Algorithm}
The main idea behind this algorithm is to rotate a vector around a point
\textit{n} times.

Following a detailed description of the algorithm:

\begin{enumerate}
  \item calculate the parameters
  \item from 1 .. node\_counts calculate the points
  \item reverse the order of points
\end{enumerate}

\subsubsection{Implementation description}
The calculation of the parameters is the first part to do. The
parameter characterize the regular polygon about the position on the
sampling grid, the radius, the winding number and the node count.
There is room for further improvements or more parameters, like to
set the segment length and calculate the possible radius and the
center.

Trigonometric functions like Sinus and Cosinus helps to calculate the points.

\begin{lstlisting}[basicstyle=\scriptsize]
  auto gamma = 2 * \pi * winding_number / node_count;
  auto rotation_angle = -gamma / 2;

  auto x = radius * std::cos(seg_counter * gamma + rotation_angle) + center.x();
  auto y = radius * std::sin(seg_counter * gamma + rotation_angle) + center.y();
\end{lstlisting}

The rotation angle is not necessary for the algorithm. But it helps to perform
calculations during other algorithms easier. The rotation angle $ - gamma / 2$
rotates the whole polygon in a way, that the first segment lies vertically.

The final step revers the order of the points. This is also not really necessary
for the algorithm, but it produces a polygon which has the same direction as the
polygons from all other algorithms.

\subsubsection{Complexity}
The $n$ in the complexity analysis stands for the node count.

\begin{enumerate}
  \item calculate the parameters $\bigO(1)$
  \item from 1 .. node\_counts calculate the points $\bigO(n)$
  \item reverse the order of points $\bigO(n)$
\end{enumerate}

This leads to the sum of $\bigO(1) + \bigO(n) + \bigO(n) \Rightarrow \bigO(n)$


\subsubsection{Parameters}
\begin{description}
  \item [--nodes] how many nodes the polygon has to have. [default: 100]
  \item [--sampling-grid] the area within the polygon could grow. [default: 1500x800]
  %% \item [--winding-number]
  \item [--segment-length] set the segment length. [default: 0]
  \item [--radius] the radius for regular polygon. [default: 60]
\end{description}

\subsubsection{Examples}
\begin{figure}[ht]
  \centering

  \begin{minipage}[t]{0.4\textwidth}
    \begin{tikzpicture}[yscale=0.05,xscale=0.05]

      \setcounter{i}{1}

      \draw[->] (0,0) -- (80,0) node[below] {$x$};
      \draw[->] (0,0) -- (0,70) node[left] {$y$};

      \foreach \p in {(40,80), (70,58), (70, 22), (40,0), (10,22), (10,58)} {
        \node[point] (\arabic{i}) at \p {};
        \stepcounter{i}
      }

      \draw (1) -- (2) -- (3) -- (4) -- (5) -- (6) -- (1);
    \end{tikzpicture}
    \caption{Regular polygon with 6 points}
    \label{fig:rp:points-6}
  \end{minipage}\hfill
  \begin{minipage}[t]{0.4\textwidth}
    \begin{tikzpicture}[yscale=0.05,xscale=0.05]

      \setcounter{i}{1}

      \draw[->] (0,0) -- (80,0) node[below] {$x$};
      \draw[->] (0,0) -- (0,70) node[left] {$y$};

      \foreach \p in {(33,70), (70,60), (70, 20), (33,10), (0,40)} {
        \node[point] (\arabic{i}) at \p {};
        \stepcounter{i}
      }

      \draw (1) -- (2) -- (3) -- (4) -- (5) -- (1);
    \end{tikzpicture}
    \caption{Regular polygon with 5 points}
    \label{fig:rp:points-5}
  \end{minipage}
\end{figure}

\FloatBarrier
