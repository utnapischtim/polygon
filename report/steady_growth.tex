\subsection{Steady Growth}
\subsubsection{Description of the Algorithm}
This algorithm is a variation of the convex hull principles. It uses
the convex hull to guarantee that no intersection occurs.
\\[12pt]
Following the description of the algorithm in detail:

\begin{enumerate}
  \item choose two points
  \item initialize polygon and convex hull
  \item randomly iterate over all points
  \begin{enumerate}
    \item the random point has to fulfil one requirement. No remaining, not
      choosen point is allowed to be located within the temporarly extended
      convex hull \fref{sg:point-p}
    \item choose a segment $s$ visible from the point $p$
    \item replace the segment $s$ with the point $p$
    \item add the point $p$ to the convex hull
  \end{enumerate}
\end{enumerate}

\subsubsection{Implementation description}
The above description of the algorithm describes only the theoretical
version of the algorithm. I would like to rewrite the algorithm in a
technical way to have a better entrance to the further description of
the implementation.

\begin{enumerate}
  \item \label{en:sg:a} two points have to be independent from each other
    \fref{sg:line}. They should not lie to far from each other. And they should
    not have another point on the line between them.
  \item \label{en:sg:b} initialize polygon and convex hull
  \item \label{en:sg:c} randomly iterate over all points
  \begin{enumerate}
    \item \label{en:sg:d} first we search the support vertices on the convex
      hull \fref{sg:hull-cv1} for the randomly choosen point \fref{sg:point-p}.
      Then it is necessary to iterate over all points that are not in the
      just constructed polygon. This iteration is necessary to verify that no
      other point lies within the temporarly created convex hull from the old
      convex hull plus the new point. If there is a point within, the support
      vertices has to be recalculated. If a point is found it is
      not necessary to iterate over the already iterated points, because these
      points could only lie outside of the new area.
    \item \label{en:sg:e}to choose a visible segment it is necessary
      to calculate a list of visible segments. First there were
      located all points between the support vertices. With this
      points and the point $p$ it is created a list of segments that
      form a closed polygon. This polygon is necessary to compute the
      visible points from the point $p$. This computation leads to a
      list of points. This points \fref{sg:reg-points} and the points
      calculated before \fref{sg:prob-vis-points} are merged to get
      segments which are completelly visible from point $p$ \fref{sg:vis-seg}.
      This list is used to choose a visible segment.
    \item \label{en:sg:f}for the replacement it is helpfull to know the segment
      on the polygon
    \item \label{en:sg:g}finally the point $p$ is added to the convex hull
  \end{enumerate}
\end{enumerate}

For \eref{sg:a} it is necessary to keep the points in a clockwise
orientation. To ensure this, there is a check if the x value from the
left point is lower than from the right one. This prevents bugs,
because the whole algorithm works only if we start with a clockwise
orientation.

\eref{sg:b} and \eref{sg:c} are implemented straight forward with a
pointlist initialization and a for-loop.

The \eref{sg:d} was difficult. It seems easy but it wasn't. The idea
is to start from the nearest point on the convex hull to the point $p$.
Then iterate right to the point where the polygon from $p$ to $u$ to $v$
doesn't form a right turn. The same in the left direction, where the
polygon from $p$ to $u$ to $v$ should form a left turn. A special case is if the
convex hull consists only of two elements. This is shorter, but the
left turn check has to be done, because of the clockwise direction. If
the check is not done, than the left and right support vertices are
assigned possibly in the wrong order. The nearest point is found by an
algorithmus that checks the intersection between the segment from the
point $p$ to the first point on the convex hull and moving segment from
first point of the hull to the second point of the hull. If such a
intersection occurs than the target is the nearest point. The first
point is used for testing purpose possibility. With the two found
support vertices it is not finished. For all points on the remaining
point list it has the be checked, that no other point lies within the
triangle of left and right support vertices and point $p$. To check this
the cgal function \textit{has\_on\_bounded\_side} from the triangle
class was used. If a point exists then the support
vertices have to be recalculated. This part could be done with minimal
effort, because it is only necessary to iterate forward from the left
support vertices to check against a right turn and from right support
vertices iterate backward and check against left turn. This could be
done in $\bigO{n}$ because processed points could only lie outside of
the possible new triangle.

The main part of \eref{sg:e} is done by the visibility algorithm
implemented by cgal. This algorithm returns a list of points that are
on the polygon and visible from point $p$. Those points does not
necessarily correspond with the points on the polygon. To get the real
visible segments it is necessary to iterate over both point lists.
Three cases exists.

\begin{enumerate}
  \item the next reg (+1) could be a visible point. or
    that the segment is on the line from $p$ to the point $p+1$
  \item after next reg (+2) could be the visible point
  \item after after next reg (+3) could be equal to a visible point
\end{enumerate}

\subsubsection{Complexity}
The complexity is $\bigO(n^2)$

\subsubsection{Parameters}
\begin{description}
  \item [--nodes] how many nodes the polygon has to have. [default: 100]
  \item [--sampling-grid] the area within the polygon could grow. [default: 1500x800]
\end{description}

\subsubsection{Examples}
\begin{figure}[ht]
  \centering

  \begin{minipage}[t]{0.4\textwidth}
    \begin{tikzpicture}[yscale=0.05,xscale=0.05]

      \setcounter{i}{0}

      \draw[->] (0,0) -- (80,0) node[below] {$x$};
      \draw[->] (0,0) -- (0,70) node[left] {$y$};

      \foreach \p in {(1,11),(5,40),(10,20),(21,10),(20,18),(20,41),
        (33,15),(39,22),(48,9),(55,17),(60,50),(68,19)} {
        \node[point] (\arabic{i}) at \p {};
        \stepcounter{i}
      }
    \end{tikzpicture}
    \caption{Base random point cloud}
    \label{fig:sg:base}
  \end{minipage}\hfill
  \begin{minipage}[t]{0.4\textwidth}
    \begin{tikzpicture}[yscale=0.05,xscale=0.05]

      \setcounter{i}{1}

      \draw[->] (0,0) -- (80,0) node[below] {$x$};
      \draw[->] (0,0) -- (0,70) node[left] {$y$};

      \foreach \p in {(1,11),(5,40),(10,20),(21,10),(20,18),(20,41),
        (33,15),(39,22),(48,9),(55,17),(60,50),(68,19)} {
        \node[point] (\arabic{i}) at \p {};
        \stepcounter{i}
      }

      \draw (7) -- (8);
    \end{tikzpicture}
    \caption{Two independent points}
    \label{fig:sg:line}
  \end{minipage}
\end{figure}

\begin{figure}[ht]
  \centering

  \begin{minipage}[t]{0.4\textwidth}
    \begin{tikzpicture}[yscale=0.05,xscale=0.05]

      \setcounter{i}{1}

      \draw[->] (0,0) -- (80,0) node[below] {$x$};
      \draw[->] (0,0) -- (0,70) node[left] {$y$};

      \foreach \p in {(1,11),(5,40),(10,20),(21,10),(20,18),(20,41),
        (33,15),(39,22),(48,9),(55,17),(60,50),(68,19)} {
        \node[point] (\arabic{i}) at \p {};
        \stepcounter{i}
      }

      \draw (7) -- (8);
      \node at (7) [below] {$s_r$};
      \node at (8) [above] {$s_l$};
      \node at (9) [right] {$p$};
    \end{tikzpicture}
    \caption{Point $p$}
    \label{fig:sg:point-p}
  \end{minipage}\hfill
  \begin{minipage}[t]{0.4\textwidth}
    \begin{tikzpicture}[yscale=0.05,xscale=0.05]

      \setcounter{i}{1}

      \draw[->] (0,0) -- (80,0) node[below] {$x$};
      \draw[->] (0,0) -- (0,70) node[left] {$y$};

      \foreach \p in {(1,11),(5,40),(10,20),(21,10),(20,18),(20,41),
        (33,15),(39,22),(48,9),(55,17),(60,50),(68,19)} {
        \node[point] (\arabic{i}) at \p {};
        \stepcounter{i}
      }

      \draw (7) -- (8) -- (9) -- (7);
    \end{tikzpicture}
    \caption{Point $p$ added}
    \label{fig:sg:point-p-add}
  \end{minipage}
\end{figure}

\begin{figure}[ht]
  \centering

  \begin{minipage}[t]{0.4\textwidth}
    \begin{tikzpicture}[yscale=0.05,xscale=0.05]

      \setcounter{i}{1}

      \draw[->] (0,0) -- (80,0) node[below] {$x$};
      \draw[->] (0,0) -- (0,70) node[left] {$y$};

      \foreach \p in {(1,11),(5,40),(10,20),(20,18),(20,41),(21,10),
        (33,15),(39,22),(48,9),(55,17),(60,50),(68,19)} {
        \node[point] (\arabic{i}) at \p {};
        \stepcounter{i}
      }

      \draw (1) -- (3) -- (4) -- (8) -- (7) -- (10) -- (12) -- (9) -- (6) -- (1);
    \end{tikzpicture}
    \caption{Polygon PY1}
    \label{fig:sg:polygon-py1}
  \end{minipage}\hfill
  \begin{minipage}[t]{0.4\textwidth}
    \begin{tikzpicture}[yscale=0.05,xscale=0.05]

      \setcounter{i}{1}

      \draw[->] (0,0) -- (80,0) node[below] {$x$};
      \draw[->] (0,0) -- (0,70) node[left] {$y$};

      \foreach \p in {(1,11),(5,40),(10,20),(20,18),(20,41),(21,10),
        (33,15),(39,22),(48,9),(55,17),(60,50),(68,19)} {
        \node[point] (\arabic{i}) at \p {};
        \stepcounter{i}
      }

      \draw (1) -- (3) -- (8) -- (12) -- (9) -- (1);
      \draw[green] (1) -- (5) -- (12);
      \node at (1) [left] {$s_l$};
      \node at (12) [right] {$s_r$};
      \node at (5) [right] {$p$};
    \end{tikzpicture}
    \caption{Convex Hull CV1 + support vertices}
    \label{fig:sg:hull-cv1}
  \end{minipage}
\end{figure}

\begin{figure}[ht]
  \centering

  \begin{minipage}[t]{0.4\textwidth}
    \begin{tikzpicture}[yscale=0.05,xscale=0.05]

      \setcounter{i}{1}

      \draw[->] (0,0) -- (80,0) node[below] {$x$};
      \draw[->] (0,0) -- (0,70) node[left] {$y$};

      \foreach \p in {(1,11),(5,40),(10,20),(20,18),(20,41),(21,10),
        (33,15),(39,22),(48,9),(55,17),(60,50),(68,19)} {
        \node[point] (\arabic{i}) at \p {};
        \stepcounter{i}
      }

      \draw[red] (1) -- (3) -- (4) -- (8) -- (7) -- (10) -- (12);
    \end{tikzpicture}
    \caption{Probably visible points}
    \label{fig:sg:prob-vis-points}
  \end{minipage}\hfill
  \begin{minipage}[t]{0.4\textwidth}
    \begin{tikzpicture}[yscale=0.05,xscale=0.05]

      \setcounter{i}{1}

      \draw[->] (0,0) -- (80,0) node[below] {$x$};
      \draw[->] (0,0) -- (0,70) node[left] {$y$};

      \foreach \p in {(1,11),(5,40),(10,20),(20,18),(20,41),(21,10),
        (33,15),(39,22),(48,9),(55,17),(60,50),(68,19)} {
        \node[point] (\arabic{i}) at \p {};
        \stepcounter{i}
      }

      \node[point, orange] (13) at (46,16) {};

      \draw[orange] (1) -- (3) -- (4) -- (8) -- (13) -- (10) -- (12);

    \end{tikzpicture}
    \caption{Regular Points}
    \label{fig:sg:reg-points}
  \end{minipage}
\end{figure}



\begin{figure}[ht]
  \centering

  \begin{minipage}[t]{0.4\textwidth}
    \begin{tikzpicture}[yscale=0.05,xscale=0.05]

      \setcounter{i}{1}

      \draw[->] (0,0) -- (80,0) node[below] {$x$};
      \draw[->] (0,0) -- (0,70) node[left] {$y$};

      \foreach \p in {(1,11),(5,40),(10,20),(20,18),(20,41),(21,10),
        (33,15),(39,22),(48,9),(55,17),(60,50),(68,19)} {
        \node[point] (\arabic{i}) at \p {};
        \stepcounter{i}
      }

      \draw (1) -- (3) -- (8) -- (12) -- (9) -- (1);
      \draw[red] (1) -- (3) -- (4) -- (8);
      \draw[red] (10) -- (12);
      \draw[green] (1) -- (5) -- (12);
      \node at (1) [left] {$s_l$};
      \node at (12) [right] {$s_r$};
      \node at (5) [right] {$p$};
    \end{tikzpicture}
    \caption{Complete Visible Segments (red) from Point $p$}
    \label{fig:sg:vis-seg}
  \end{minipage}\hfill
  \begin{minipage}[t]{0.4\textwidth}
    \begin{tikzpicture}[yscale=0.05,xscale=0.05]

      \setcounter{i}{1}

      \draw[->] (0,0) -- (80,0) node[below] {$x$};
      \draw[->] (0,0) -- (0,70) node[left] {$y$};

      \foreach \p in {(1,11),(5,40),(10,20),(20,18),(20,41),(21,10),
        (33,15),(39,22),(48,9),(55,17),(60,50),(68,19)} {
        \node[point] (\arabic{i}) at \p {};
        \stepcounter{i}
      }

      \draw (1) -- (3) -- (4) --(5) -- (8) -- (7) -- (10) -- (12) -- (9) -- (6) -- (1);

    \end{tikzpicture}
    \caption{Polygon PY2}
    \label{fig:sg:polygon-py2}
  \end{minipage}
\end{figure}

\begin{figure}[ht]
  \centering

  \begin{minipage}[t]{0.4\textwidth}
    \begin{tikzpicture}[yscale=0.05,xscale=0.05]

      \setcounter{i}{1}

      \draw[->] (0,0) -- (80,0) node[below] {$x$};
      \draw[->] (0,0) -- (0,70) node[left] {$y$};

      \foreach \p in {(1,11),(5,40),(10,20),(20,18),(20,41),(21,10),
        (33,15),(39,22),(48,9),(55,17),(60,50),(68,19)} {
        \node[point] (\arabic{i}) at \p {};
        \stepcounter{i}
      }

      \draw (1) -- (2) -- (3) -- (4) --(5) -- (8) -- (7) -- (10) -- (12) -- (9) -- (6) -- (1);
    \end{tikzpicture}
    \caption{Possible next step}
    \label{fig:sg:polygon-py3}
  \end{minipage}\hfill
  \begin{minipage}[t]{0.4\textwidth}
    \begin{tikzpicture}[yscale=0.05,xscale=0.05]

      \setcounter{i}{1}

      \draw[->] (0,0) -- (80,0) node[below] {$x$};
      \draw[->] (0,0) -- (0,70) node[left] {$y$};

      \foreach \p in {(1,11),(5,40),(10,20),(20,18),(20,41),(21,10),
        (33,15),(39,22),(48,9),(55,17),(60,50),(68,19)} {
        \node[point] (\arabic{i}) at \p {};
        \stepcounter{i}
      }

      \draw (1) -- (2) -- (3) -- (4) --(5) -- (11) -- (8) -- (7) -- (10) -- (12) -- (9) -- (6) -- (1);
    \end{tikzpicture}
    \caption{Possible last step}
    \label{fig:sg:polygon-py4}
  \end{minipage}
\end{figure}

\FloatBarrier
