\subsection{Convex Bottom}
\subsubsection{Description of the Algorithm}
Convex Bottom Algorithm uses the Convex Hull principle to create a convex
polygon. The algorithm starts from a random 2d point cloud.
\fref{cb:base}.
\\[12pt]
Following the description of the algorithm in detail:

\begin{enumerate}
  \item sort all points along the x-axis.
  \item choose the lowest and greatest point on x-axis and draw a line between them\fref{cb:line}
  \item create a list of all points below or on this line \fref{cb:below-line}
  \item create a convex hull from those points \fref{cb:hull}
  \item add points not on the convex bottom hull to the polygon \fref{cb:polygon-1}
  \item add points from the convex bottom hull in reverse order to the polygon
    \fref{cb:polygon-2}
\end{enumerate}

\subsubsection{Implementation description}
\begin{enumerate}
  \item the function std::sort from the std lib does the sorting
  \item the sorted point list gives as the possibility to use the first point
    and the last point as the point with the lowest and the highest x value
  \item the CGAL::orientation function in combination with a for loop checks if
    the point lies below the formed line
  \item the CGAL::convex\_hull\_2 creates the bottom convex hull
  \item add all points which does not lie on the convex bottom hull to the polygon
  \item with std::reverse and a for loop the points from the convex bottom hull
    were added to the final polygon
\end{enumerate}

\subsubsection{Complexity}
\begin{enumerate}
  \item sort all points along the x-axis. $\bigO(nlogn)$
  \item choose the lowest and greatest point on x-axis and draw a line between them $\bigO(1)$
  \item create a list of all points below or on this line $\bigO(n)$
  \item create a convex hull from those points $\bigO(nlogn)$
  \item add points not on the convex bottom hull to the polygon $\bigO(n)$
  \item add points from the convex bottom hull in reverse order to the polygon $\bigO(n)$
\end{enumerate}
In summarize this leads to $\bigO(nlogn)$.

\subsubsection{Parameters}
\begin{description}
  \item [--nodes] how many nodes the polygon have to have. [default: 100]
  \item [--sampling-grid] the area within the polygon could grow. [default: 1500x800]
\end{description}

\subsubsection{Examples}

\begin{figure}[ht]
  \centering

  \begin{minipage}[t]{0.4\textwidth}
    \begin{tikzpicture}[yscale=0.05,xscale=0.05]

      \setcounter{i}{0}

      \draw[->] (0,0) -- (80,0) node[below] {$x$};
      \draw[->] (0,0) -- (0,70) node[left] {$y$};

      \foreach \p in {(1,11),(5,40),(10,20),(21,10),(20,18),(20,41),(33,15),(39,22),(48,9),(55,17),(60,50),(68,19)} {
        \node[point] (\arabic{i}) at \p {};
        \stepcounter{i}
      }
    \end{tikzpicture}
    \caption{Base random point cloud}
    \label{fig:cb:base}
  \end{minipage}\hfill
  \begin{minipage}[t]{0.4\textwidth}
    \begin{tikzpicture}[yscale=0.05,xscale=0.05]

      \setcounter{i}{1}

      \draw[->] (0,0) -- (80,0) node[below] {$x$};
      \draw[->] (0,0) -- (0,70) node[left] {$y$};

      \foreach \p in {(1,11),(5,40),(10,20),(21,10),(20,18),(20,41),(33,15),(39,22),(48,9),(55,17),(60,50),(68,19)} {
        \node[point] (\arabic{i}) at \p {};
        \stepcounter{i}
      }

      \draw (1) -- (12);
    \end{tikzpicture}
    \caption{Line from lowest x to highest x}
    \label{fig:cb:line}
  \end{minipage}
\end{figure}

\begin{figure}[ht]
  \centering
  \begin{minipage}[t]{0.4\textwidth}
    \begin{tikzpicture}[yscale=0.05,xscale=0.05]

      \draw[->] (0,0) -- (80,0) node[below] {$x$};
      \draw[->] (0,0) -- (0,70) node[left] {$y$};

      \node[point, fill=red] (1) at (1,11) {};
      \node[point] (2) at (5,40) {};
      \node[point] (3) at (10,20) {};
      \node[point, fill=red] (4) at (21,10) {};
      \node[point] (5) at (20,18) {};
      \node[point] (6) at (20,41) {};
      \node[point] (7) at (33,15) {};
      \node[point] (8) at (39,22) {};
      \node[point, fill=red] (9) at (48,9) {};
      \node[point, fill=red] (10) at (55,17) {};
      \node[point] (11) at (60,50) {};
      \node[point, fill=red] (12) at (68,19) {};

      \draw (1) -- (12);
    \end{tikzpicture}
    \caption{Points on and below of the line}
    \label{fig:cb:below-line}
  \end{minipage}\hfill
  \begin{minipage}[t]{0.4\textwidth}
    \centering
    \begin{tikzpicture}[yscale=0.05,xscale=0.05]

      \setcounter{i}{1}

      \draw[->] (0,0) -- (80,0) node[below] {$x$};
      \draw[->] (0,0) -- (0,70) node[left] {$y$};

      \foreach \p in {(1,11),(5,40),(10,20),(21,10),(20,18),(20,41),(33,15),(39,22),(48,9),(55,17),(60,50),(68,19)} {
        \node[point] (\arabic{i}) at \p {};
        \stepcounter{i}
      }

      \draw (1) -- (12) -- (9) -- (4) -- (1);
    \end{tikzpicture}
    \caption{Convex bottom hull}
    \label{fig:cb:hull}
  \end{minipage}
\end{figure}

\begin{figure}[ht]
  \centering
  \begin{minipage}[t]{0.4\textwidth}
    \begin{tikzpicture}[yscale=0.05,xscale=0.05]
      \setcounter{i}{1}

      \draw[->] (0,0) -- (80,0) node[below] {$x$};
      \draw[->] (0,0) -- (0,70) node[left] {$y$};

      \foreach \p in {(1,11),(5,40),(10,20),(21,10),(20,18),(20,41),(33,15),(39,22),(48,9),(55,17),(60,50),(68,19)} {
        \node[point] (\arabic{i}) at \p {};
        \stepcounter{i}
      }

      \draw (1) -- (2) -- (3) -- (5) -- (6) -- (7) -- (8) -- (10) -- (11) -- (12);
    \end{tikzpicture}
    \caption{Polygon from the sorted values}
    \label{fig:cb:polygon-1}
  \end{minipage}\hfill
  \begin{minipage}[t]{0.4\textwidth}
    \centering
    \begin{tikzpicture}[yscale=0.05,xscale=0.05]

      \setcounter{i}{1}

      \draw[->] (0,0) -- (80,0) node[below] {$x$};
      \draw[->] (0,0) -- (0,70) node[left] {$y$};

      \foreach \p in {(1,11),(5,40),(10,20),(21,10),(20,18),(20,41),(33,15),(39,22),(48,9),(55,17),(60,50),(68,19)} {
        \node[point] (\arabic{i}) at \p {};
        \stepcounter{i}
      }
      \draw (1) -- (2) -- (3) -- (5) -- (6) -- (7) -- (8) -- (10) -- (11) -- (12) -- (9) -- (4) -- (1);
    \end{tikzpicture}
    \caption{Polygon from above combined with polygon from below}
    \label{fig:cb:polygon-2}
  \end{minipage}
\end{figure}

\FloatBarrier
