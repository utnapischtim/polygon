\subsection{Random Two Peasants}
\subsubsection{Description of the Algorithm}
The main idea for the random two peasants algorithm is to sort the points along
a axis. The algorithm start from a random 2d point cloud \fref{rtp:base}

\begin{enumerate}
  \item sort all points along the x-axis.
  \item use the line \fref{rtp:line} from lowest point to greatest point
    on x-axis to divide the points in a upper \fref{rtp:upper} and a lower
    \fref{rtp:lower} point cloud.
  \item add sequential all points from the upper list to the polygon
    \fref{rtp:polygon-upper}
  \item add in reverse order all points from the lower list to the
    polygon \fref{rtp:polygon}
\end{enumerate}

\subsubsection{Implementation description}

\begin{enumerate}
  \item the function std::sort from the std lib does the sorting
  \item iterate over all points and perform the CGAL::orientation function to
    get the orientation of all points aligned at the line between the lowest
    point and the highest point.
  \item iterate over all points from the upper list and add it to the polygon
  \item iterate over all points from the lower list in reverse direction and add
    it to the polygon.
\end{enumerate}

\subsubsection{Complexitiy}

\begin{enumerate}
  \item sort all points along the x-axis. $\bigO(nlogn)$
  \item use the line from lowest point to greatest point on x-axis to divide the
    points in a upper and a lower point cloud. $\bigO(n)$
  \item add sequential all points from the upper list to the polygon $\bigO(n)$
  \item add in reverse order all points from the lower list to the polygon $\bigO(n)$
\end{enumerate}

This leads to $max(\bigO(nlogn) + \bigO(n) + \bigO(n) + \bigO(n) \Rightarrow \bigO(nlogn))$.

\subsubsection{Parameters}
\begin{description}
  \item [--nodes] how many nodes the polygon has to have. [default: 100]
  \item [--sampling-grid] the area within the polygon could grow. [default: 1500x800]
\end{description}

\subsubsection{Examples}

\begin{figure}[ht]
  \centering

  \begin{minipage}[t]{0.4\textwidth}
    \begin{tikzpicture}[yscale=0.05,xscale=0.05]
      \setcounter{i}{1}

      \draw[->] (0,0) -- (80,0) node[below] {$x$};
      \draw[->] (0,0) -- (0,70) node[left] {$y$};

      \foreach \p in {(1,11),(5,40),(10,20),(21,10),(20,18),(20,41),(33,15),(39,22),(48,9),(55,17),(60,50),(68,19)} {
        \node[point] (\arabic{i}) at \p {};
        \stepcounter{i}
      }
    \end{tikzpicture}
    \caption{Base random point cloud}
    \label{fig:rtp:base}
  \end{minipage}\hfill
  \begin{minipage}[t]{0.4\textwidth}
    \begin{tikzpicture}[yscale=0.05,xscale=0.05]
      \setcounter{i}{1}

      \draw[->] (0,0) -- (80,0) node[below] {$x$};
      \draw[->] (0,0) -- (0,70) node[left] {$y$};

      \foreach \p in {(1,11),(5,40),(10,20),(21,10),(20,18),(20,41),(33,15),(39,22),(48,9),(55,17),(60,50),(68,19)} {
        \node[point] (\arabic{i}) at \p {};
        \stepcounter{i}
      }

      \draw[blue] (1) -- (12);
    \end{tikzpicture}
    \caption{Dividing Line}
    \label{fig:rtp:line}
  \end{minipage}
\end{figure}

\begin{figure}[ht]
  \centering

  \begin{minipage}[t]{0.4\textwidth}
    \begin{tikzpicture}[yscale=0.05,xscale=0.05]
      \setcounter{i}{1}

      \draw[->] (0,0) -- (80,0) node[below] {$x$};
      \draw[->] (0,0) -- (0,70) node[left] {$y$};

      \foreach \p in {(1,11),(5,40),(10,20),(20,18),(20,41),(33,15),(39,22),(60,50),(68,19)} {
        \node[point] (\arabic{i}) at \p {};
        \stepcounter{i}
      }

      \foreach \p in {(21,10),(48,9),(55,17)} {
        \node[point, fill=red] (\arabic{i}) at \p {};
        \stepcounter{i}
      }

      \draw[blue] (1) -- (9);
    \end{tikzpicture}
    \caption{Lower List}
    \label{fig:rtp:lower}
  \end{minipage}\hfill
  \begin{minipage}[t]{0.4\textwidth}
    \begin{tikzpicture}[yscale=0.05,xscale=0.05]
      \setcounter{i}{1}

      \draw[->] (0,0) -- (80,0) node[below] {$x$};
      \draw[->] (0,0) -- (0,70) node[left] {$y$};

      \foreach \p in {(1,11),(21,10),(48,9),(55,17),(68,19)} {
        \node[point] (\arabic{i}) at \p {};
        \stepcounter{i}
      }

      \foreach \p in {(5,40),(10,20),(20,18),(20,41),(33,15),(39,22),(60,50)} {
        \node[point, fill=red] (\arabic{i}) at \p {};
        \stepcounter{i}
      }

      \draw[blue] (1) -- (5);
    \end{tikzpicture}
    \caption{Upper list}
    \label{fig:rtp:upper}
  \end{minipage}
\end{figure}

\begin{figure}[ht]
  \centering

  \begin{minipage}[t]{0.4\textwidth}
    \begin{tikzpicture}[yscale=0.05,xscale=0.05]
      \setcounter{i}{1}

      \draw[->] (0,0) -- (80,0) node[below] {$x$};
      \draw[->] (0,0) -- (0,70) node[left] {$y$};

      \foreach \p in {(1,11),(5,40),(10,20),(21,10),(20,18),(20,41),(33,15),(39,22),(48,9),(55,17),(60,50),(68,19)} {
        \node[point] (\arabic{i}) at \p {};
        \stepcounter{i}
      }
      \draw (1) -- (2) -- (3) -- (5) -- (6) -- (7) -- (8) -- (11) -- (12);
    \end{tikzpicture}
    \caption{The Upper polygon}
    \label{fig:rtp:polygon-upper}
  \end{minipage}\hfill
  \begin{minipage}[t]{0.4\textwidth}
    \begin{tikzpicture}[yscale=0.05,xscale=0.05]
      \setcounter{i}{1}

      \draw[->] (0,0) -- (80,0) node[below] {$x$};
      \draw[->] (0,0) -- (0,70) node[left] {$y$};

      \foreach \p in {(1,11),(5,40),(10,20),(21,10),(20,18),(20,41),(33,15),(39,22),(48,9),(55,17),(60,50),(68,19)} {
        \node[point] (\arabic{i}) at \p {};
        \stepcounter{i}
      }

      \draw (1) -- (2) -- (3) -- (5) -- (6) -- (7) -- (8) -- (11) -- (12) -- (10) -- (9) -- (4) -- (1);
    \end{tikzpicture}
    \caption{Final polygon}
    \label{fig:rtp:polygon}
  \end{minipage}
\end{figure}

\FloatBarrier
