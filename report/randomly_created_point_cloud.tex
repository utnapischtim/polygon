\subsection{Randomly Created Point Cloud}
\subsubsection{Description of the Algorithm}
The algorithm creates a point cloud where the x and y values for every point
where calculated by a random function.

\subsubsection{Implementation description}
The implementation uses the std::uniform\_real\_distribution function from the
c++ standard. It creates values from 0 to width and 0 to height.

\subsubsection{Complexity}
The complexity is $\bigO(n)$ where $n$ is the number of point count.

\subsubsection{Parameters}
\begin{description}
  \item [--nodes] how many nodes the polygon has to have. [default: 100]
  \item [--sampling-grid] the area within the polygon could grow. [default: 1500x800]
\end{description}


\subsubsection{Examples}
\begin{figure}[ht]
  \centering

  \begin{minipage}[t]{0.4\textwidth}
    \begin{tikzpicture}[yscale=0.05,xscale=0.05]

      \setcounter{i}{1}

      \draw[->] (0,0) -- (80,0) node[below] {$x$};
      \draw[->] (0,0) -- (0,70) node[left] {$y$};

      \foreach \p in {(18, 7),(1, 39),(77, 48),(56, 13),(77, 51),(56, 27),(61, 9),(56, 12),(69, 63),(45, 53)} {
        \node[point] (\arabic{i}) at \p {};
        \stepcounter{i}
      }

      \draw (1) -- (2) -- (3) -- (4) -- (5) -- (6) -- (7) -- (8) -- (9) -- (10)
      -- (1);
    \end{tikzpicture}
    \caption{Example point cloud used as a polygon with 10 points}
    \label{fig:rcpc:points-10}
  \end{minipage}\hfill
  \begin{minipage}[t]{0.4\textwidth}
    \begin{tikzpicture}[yscale=0.05,xscale=0.05]

      \setcounter{i}{1}

      \draw[->] (0,0) -- (80,0) node[below] {$x$};
      \draw[->] (0,0) -- (0,70) node[left] {$y$};

      \foreach \p in {(70,60), (33,70), (33,10), (70, 20), (0,40)} {
        \node[point] (\arabic{i}) at \p {};
        \stepcounter{i}
      }

      \draw (1) -- (2) -- (3) -- (4) -- (5) -- (1);
    \end{tikzpicture}
    \caption{Example point cloud used as a polygon with 5 points}
    \label{fig:rcpc:points-2}
  \end{minipage}
\end{figure}

\FloatBarrier
