\subsection{Bouncing Vertices}
\subsubsection{Description of the algorithm}
The main idea behind Bouncing Vertices is to take a simple polygon and
bounce the points randomly by retaining the simple status. Further, it
is possible to restrict the movement of the bouncing step by certain
requirements. Requirements could be, for example to retain the convex
and/or reflex status of the bouncing point.

\subsubsection{Steps of the algorithm}

The bouncing loop could be achieved in two different ways. Both have been
implemented.

\paragraph{Check after the point is temporarly moved}
\begin{enumerate}
  \item initialize settings
  \item convert point list to segment list
  \item bounce every point n times
  \begin{enumerate}
    \item choose a point inside the sampling grid
    \item if wanted, check if the orientation is still the same
    \item if wanted, check if the angle is still within the given range
    \item check if the moved point causes an intersection
    \item follow the above steps until a point fullfills all of them
  \end{enumerate}
\end{enumerate}

\paragraph{Calculate the allowed segment before moving the point}
\begin{enumerate}
  \item initialize settings
  \item convert point list to segment list
  \item bounce every point n times
  \begin{enumerate}
    \item choose a random angle to move the point
    \item calculate the random line to move the point
    \item calculate allowed segment which is has no intersections with other
      segments or the border
    \item if wanted, calculate allowed segment to keep the angle range for the
      current point and the point before and the point after the current point
    \item if wanted, calculate allowed segment to keep the orientation of the
      current point, the point before and the point after the current point
    \item merge those allowed segments to the smallest allowed segment
    \item choose a random value [0,distance)
    \item move the point to the new position
  \end{enumerate}
\end{enumerate}

\subsubsection{Description of implementation}
This algorithm is straight forward to implement, and also gives the
opportunity to extend the requirements to bounce a point, like the
angle range or the orientation stability.

As mentioned above, the bouncing loop can be done in two different
ways. Both possibilities of implementation are described below.

\paragraph{Check after the point is temporarly moved}

\begin{enumerate}
  \item the initialization of the values succeeds by the
    ``BouncingVerticesSettings'' struct
  \item the conversion from point list to segment list is implemented
    with a ``for loop''
  \item bounce every point n times
  \begin{enumerate}
    \item choose a velocity vector randomly within the range given
      by the bouncing radius parameter within the sampling grid
      boundaries. This is implemented with a do while loop.
    \item to check the orientation stability it is necessary to
      compare the old and the new segments around the bouncing point.
      Three equalities have to be equal: the orientation in the
      bouncing point, the orientation in the point before and the
      orientation in the point after the bouncing point. This is
      necessary because the current point affects the orientation of
      the previous point and next point.
    \item there are two cases to consider when the angle range needs
      to be checked. The first is, if the orientation takes place,
      then the old points have to be used to calculate the necessary
      angle influenced by the old orientation. The second case is, if
      the orientation does not take place, and therefore only the
      angle range is used. To implement this the current segments
      were used to calculate the orientation to choose the necessary
      angle range.
    \item the intersection check is also straight forward implemented.
      a loop to iterate over all segments, and to check if both new
      segments would create a intersection. Some cases have to be
      noted. The checked segment should be different from the old
      segments. If the current segment to check has common starting
      points with the new segments, then it is only checked about
      collinearity which would be interpreted as a intersection.
  \end{enumerate}
\end{enumerate}

\paragraph{Calculate the allowed segment before moving the point}
\begin{enumerate}
  \item the initialization of the values succeeds by the
    BouncingVerticesSettings struct
  \item the convertion from point list to segment list is straight
    forward a for loop
  \item bounce every segment n times
  \begin{enumerate}
    \item the value gets choosen by a random function from [0, 360).
      the whole range could be used except of two values. The angles
      from the next and the previous segment.
    \item create a temporary line and rotate that around the current point
    \item calculate the intersection with the sampling\_grid. then iterate over
      all segments and check if there are intersections and if the distance to
      the bouncing point is lower. Update the allowed segment if necessary
    \item to keep the angle range it is necessary to do two steps. First the
      allowed segment for the combination of the point before and the point
      after the current point has to be calculated. This is done by rotating the
      line throw the segment before by the angle range and intersect it with the
      random line. This produces 4 allowed segments. These have to be combined
      to two according to the orientation of the points before and after. The
      second step is to calculate the allowed segment according to the angle
      range of the current point. To do this the Thales's Theorem is used. The
      first step is to calculate the bisecting line of the point before A and the
      point after B the current point C. The next step is to create a vector
      from A to B. This vector is rotated according to the angle range. The
      intersection between this vector and the bisecting line is the center of
      the circle in which every line from point A to the circle to point B has
      the angle with which the vector was rotated.
    \item to calculate the allowed segment for the orientation stability it is
      necessary to intersect the prev segment (target is A) and the next segment
      (source is A) with the random line and construct a allowed segment.
    \item merge the allowed segments to find the smallest allowed
    \item calculate the new point by choosing a value $d$ between 0 and distance of
      allowed segment $D$. calculate the point by $d/D * A + (1-d/D) * B$
  \end{enumerate}
\end{enumerate}


\subsubsection{Complexity}
\paragraph{Check after the point is temporarly moved}
\begin{enumerate}
  \item initialize settings $\bigO(1)$
  \item convert point list to segment list $\bigO(n)$
  \item bounce every segment n times $\bigO(n)$
  \begin{enumerate}
    \item choose a point inside the sampling grid $\bigO(P(x = point\ inside\ grid))$
    \item if wanted check if the orientation is still the same $\bigO(1)$
    \item if wanted check if the angle is still in given range $\bigO(1)$
    \item check if the moved point cause a intersection $\bigO(n)$
    \item do the above steps until a point fullfills all of them $\bigO(k)$
  \end{enumerate}
\end{enumerate}
The complexity depends with this implementation a little on the
probabilty how fast a correct point is found. Otherwise every point
has to be bounced for a fixed count of phases. On every iteration
every segment has to be checked of intersection, therefore
$\bigO(n^2*k)$ should be the complexity. n is the number of points, k
is the number how often a new point has to be choosen to fullfill the
requirements. This value depends on the parameter set to restrict the
valid area.

\paragraph{Calculate the allowed segment before moving the point}
\begin{enumerate}
  \item initialize settings $\bigO(1)$
  \item convert point list to segment list $\bigO(n)$
  \item bounce every segment n times $\bigO(n)$
  \begin{enumerate}
    \item all calculation have $\bigO(1)$
    \item the intersection part has $\bigO(n)$
  \end{enumerate}
\end{enumerate}

This leads to $\bigO(n^2)$

\subsubsection{Parameters}
\begin{description}
  \item [--nodes] how many nodes the polygon have to have. [default: 100]
  \item [--sampling-grid] the area within the polygon could grow. [default: 1500x800]
  \item [--phases] how many phases a random algorithm should run. [default: 10]
  \item [--bouncing-radius] the distance to move a random point from old to new point. [default: 60]
  \item [--animation] create a animation of the different phases of bouncing vertices. default is 0 for false, to start the creation set this to 1. [default: 0]
  \item [--out-every-phase] to output every phase as a gnuplot file. [default: 0]
  \item [--reflex-points] set the number of reflex points that should be at least. [default: 0]
  \item [--reflex-angle-range] set the range of the reflex angle. the angle is interpreted as open in mathematical sense (180..360). [default: 180..360]
  \item [--convex-angle-range] set the range of the convex angle. the angle is interpreted as open in mathematical sense (0..180). [default: 0..180]
\end{description}

\subsubsection{Examples}
\begin{figure}[ht]
  \centering

  \begin{minipage}[t]{0.4\textwidth}
    \begin{tikzpicture}[yscale=0.05,xscale=0.05]

      \setcounter{i}{1}

      \draw[->] (0,0) -- (80,0) node[below] {$x$};
      \draw[->] (0,0) -- (0,70) node[left] {$y$};

      \foreach \p in {(62,26),(51,11),(34,6),(16,11),(5,26),(5,45),(16,60),(34,66),(51,60),(62,45)} {
        \node[point] (\arabic{i}) at \p {};
        \stepcounter{i}
      }

      \draw (1) -- (2) -- (3) -- (4) -- (5) -- (6) -- (7) -- (8) -- (9) -- (10) -- (1);
    \end{tikzpicture}
    \caption{Regular Polygon. Startpoint for the algorithm}
    \label{fig:bv:base}
  \end{minipage}\hfill
  \begin{minipage}[t]{0.4\textwidth}
    \begin{tikzpicture}[yscale=0.05,xscale=0.05]

      \setcounter{i}{1}

      \draw[->] (0,0) -- (80,0) node[below] {$x$};
      \draw[->] (0,0) -- (0,70) node[left] {$y$};

      \foreach \p in {(62,26),(51,11),(34,6),(16,11),(5,26),(15,45),(16,60),(34,66),(51,60),(62,45)} {
        \node[point] (\arabic{i}) at \p {};
        \stepcounter{i}
      }

      \draw (1) -- (2) -- (3) -- (4) -- (5) -- (6) -- (7) -- (8) -- (9) -- (10) -- (1);
    \end{tikzpicture}
    \caption{Point 6 has moved}
    \label{fig:bv:bounce2}
  \end{minipage}
\end{figure}

\begin{figure}[ht]
  \centering

  \begin{minipage}[t]{0.4\textwidth}
    \begin{tikzpicture}[yscale=0.05,xscale=0.05]

      \setcounter{i}{1}

      \draw[->] (0,0) -- (80,0) node[below] {$x$};
      \draw[->] (0,0) -- (0,70) node[left] {$y$};

      \foreach \p in {(62,26),(30,11),(34,6),(16,11),(5,26),(15,45),(16,60),(34,66),(51,60),(62,45)} {
        \node[point] (\arabic{i}) at \p {};
        \stepcounter{i}
      }

      \draw (1) -- (2) -- (3) -- (4) -- (5) -- (6) -- (7) -- (8) -- (9) -- (10) -- (1);
    \end{tikzpicture}
    \caption{Point 3 has moved}
    \label{fig:bv:bounce3}
  \end{minipage}\hfill
  \begin{minipage}[t]{0.4\textwidth}
    \begin{tikzpicture}[yscale=0.05,xscale=0.05]

      \setcounter{i}{1}

      \draw[->] (0,0) -- (80,0) node[below] {$x$};
      \draw[->] (0,0) -- (0,70) node[left] {$y$};

      \foreach \p in {(62,26),(30,11),(34,6),(16,11),(5,26),(15,45),(16,60),(34,66),(65,56),(62,45)} {
        \node[point] (\arabic{i}) at \p {};
        \stepcounter{i}
      }

      \draw (1) -- (2) -- (3) -- (4) -- (5) -- (6) -- (7) -- (8) -- (9) -- (10) -- (1);
    \end{tikzpicture}
    \caption{Point 9 has moved}
    \label{fig:bv:bounce4}
  \end{minipage}
\end{figure}

\begin{figure}[ht]
  \centering

  \begin{minipage}[t]{0.4\textwidth}
    \begin{tikzpicture}[yscale=0.05,xscale=0.05]

      \setcounter{i}{1}

      \draw[->] (0,0) -- (80,0) node[below] {$x$};
      \draw[->] (0,0) -- (0,70) node[left] {$y$};

      \foreach \p in {(62,26),(30,11),(34,6),(16,11),(5,26),(15,45),(70,47),(34,66),(65,56),(62,45)} {
        \node[point] (\arabic{i}) at \p {};
        \stepcounter{i}
      }

      \draw (1) -- (2) -- (3) -- (4) -- (5) -- (6);
      \draw (10) -- (1);
      \draw (8) -- (9);
      \draw[red] (6) -- (7) -- (8);
      \draw[green] (9) -- (10);
    \end{tikzpicture}
    \caption{The red lines cut the green line, which is not good, and not valid!}
    \label{fig:bv:bounce5}
  \end{minipage}\hfill
  \begin{minipage}[t]{0.4\textwidth}
    \begin{tikzpicture}[yscale=0.05,xscale=0.05]

      \setcounter{i}{1}

      \draw[->] (0,0) -- (80,0) node[below] {$x$};
      \draw[->] (0,0) -- (0,70) node[left] {$y$};

      \foreach \p in {(62,26),(30,11),(34,6),(16,11),(5,26),(15,45),(16,20),(34,66),(65,56),(62,45)} {
        \node[point] (\arabic{i}) at \p {};
        \stepcounter{i}
      }

      \draw (1) -- (2) -- (3) -- (4) -- (5) -- (6) -- (7) -- (8) -- (9) -- (10) -- (1);
    \end{tikzpicture}
    \caption{Valid movement!}
    \label{fig:bv:bounce6}
  \end{minipage}
\end{figure}


\begin{figure}[ht]
  \begin{tikzpicture}[yscale=0.2,xscale=0.2]
    \setcounter{i}{1}

    \draw[->] (0,0) -- (80,0) node[below] {$x$};
    \draw[->] (0,0) -- (0,70) node[left] {$y$};

    \foreach \p in {(10,10),(30,10),(30,30),(10,30)} {
      \node[point] (\arabic{i}) at \p {};
      \stepcounter{i}
    }

    \draw (1) -- (2) -- (3) -- (4) -- (1);

    \node (5) at (0,20) {};
    \node (6) at (20,0) {};
    \draw (5) -- (6);
  \end{tikzpicture}
  \caption{first test}
\end{figure}

\newcommand{\axis}{
  \draw[->] (0,0) -- (13,0) node[below] {$x$};
  \draw[->] (0,0) -- (0,10) node[left] {$y$};
}

\newcommand{\polygon}[1]{
  \draw[name path=polygon]
    \foreach \x [count=\xi] in {#1} {
      \x node[point] (node\xi) {}
    }

    \foreach \x [count=\xi, remember=\xi-1 as \xiprev] in {#1} {
      \ifnum\xi>1%
        (node\xiprev) -- (node\xi)
      \fi
    }
  ;
}

\newcommand{\randomLine}[4]{
  \ifthenelse{\equal{#3}{}}{
    \draw[name path=randomLine,rotate=0] #1 -- #2;
  }{
    \draw[name path=randomLine,rotate around={#3:#4}] #1 -- #2;
  }
}

\newcommand{\intersection}[2]{
  \fill[red, name intersections={of=#1 and #2, total=\t}]
    \foreach \s in {1,...,\t}{(intersection-\s) circle (5pt) node{\s}};
}

\begin{figure}[ht]
  \begin{tikzpicture}[yscale=0.75,xscale=0.75]
    \axis
    \polygon{(2,2),(3,4),(9,6),(12,1),(8,2),(5,1),(2,2)}
    \randomLine{(5,0)}{(5,7)}{}{}
    \intersection{randomLine}{polygon}
  \end{tikzpicture}
  \caption{intersection free with vertical line}
\end{figure}

\begin{figure}[ht]
  \begin{tikzpicture}[yscale=0.75,xscale=0.75]
    \axis
    \polygon{(2,2),(3,4),(9,6),(12,1),(8,2),(5,1),(2,2)}
    \randomLine{(0,2)}{(13,2)}{}{}
    \intersection{randomLine}{polygon}
  \end{tikzpicture}
  \caption{intersection free with horizontal line}
\end{figure}

\begin{figure}[ht]
  \begin{tikzpicture}[yscale=0.75,xscale=0.75]
    \axis
    \polygon{(2,2),(3,4),(9,6),(12,1),(8,2),(5,1),(2,2)}
    \randomLine{(0,2)}{(13,2)}{30}{(8,2)}
    \intersection{randomLine}{polygon}
  \end{tikzpicture}
  \caption{intersection free with slightly rotated line}
\end{figure}

\begin{figure}[ht]
  \begin{tikzpicture}[yscale=0.75,xscale=0.75]
    \axis
    \polygon{(2,2),(3,4),(9,6),(12,1),(8,2),(5,1),(2,2)}
    \randomLine{(5,0)}{(5,7)}{}{}

    \draw[red, shorten >=-3cm, shorten <= -3cm] (2,2) -- (3,4);
    \draw[red, shorten >=-3cm, shorten <= -3cm] (2,2) -- (8,2);
    \draw[red, shorten >=-3cm, shorten <= -3cm] (12,1) -- (8,2);

    %% \intersection{randomLine}{reflexLine}
    %% \intersection{randomLine}{convexLine}
  \end{tikzpicture}
  \caption{orientation stability with vertical line}
\end{figure}

\begin{figure}[ht]
  \begin{tikzpicture}[yscale=0.35,xscale=0.35]
    \draw[->] (0,0) -- (40,0) node[below] {$x$};
    \draw[->] (0,0) -- (0,40) node[left] {$y$};
    \polygon{(23.7135,28.4134), (21.5339, 25.4134), (18.0072,24.2675), (14.4805,25.4134), (12.3009, 28.4134), (12.3009, 32.1216), (14.4805, 35.1216), (18.0072, 36.2675), (21.5339, 35.1216), (23.7135,32.1216),(23.7135,28.4134)}
    \randomLine{(18.5339, 25.4134)}{(38.5339, 25.4134)}{128.25}{(21.5339, 25.4134)}
    \intersection{randomLine}{polygon}
  \end{tikzpicture}
  \caption{intersection free with slightly rotated line}
\end{figure}

\begin{figure}[ht]
  \begin{tikzpicture}[yscale=0.25,xscale=0.25]
    \draw[->] (0,0) -- (70,0) node[below] {$x$};
    \draw[->] (0,0) -- (0,70) node[left] {$y$};
    \polygon{(20,30),(20,50),(30,60),(50,60),(60,50),(60,30),(28,24),(30,20),(20,30)}
    \randomLine{(-10,20)}{(70,20)}{-30}{(30,20)}
    %%\intersection{randomLine}{polygon}
    \draw[shorten >=-6cm] (20,30) -- (28,24);
  \end{tikzpicture}
  \caption{intersection free with slightly rotated line}
\end{figure}


\FloatBarrier
