\subsection{2-Opt Moves}
\subsubsection{Description of the Algorithm}
The main idea behind this algorithm is to construct the simply polygon
by resolving the intersections.

\begin{enumerate}
  \item create segments from a random polygon
  \item calculate all intersections of segments
  \item iterate over all intersections
	\begin{enumerate}
    \item choose a intersection
    \item remove all intersections where one of the two segments are
      involved
    \item remove the segments from the intersection from the segments
      list
    \item create new segments from those two segments
    \item calculate all intersection that occure by the new segments
    \item add the two segments to the segments list
  \end{enumerate}
  \item calculate the final list of the segments list
\end{enumerate}

\subsubsection{Implementation description}
\begin{enumerate}
  \item the creation of the segments is trivial
  \item the calculation of the intersections consists of two nested
    loops over all segments. The important part is to have no
    duplicates of intersections. This is implemented with a map and a
    duplicate check. The map was choosen because of the O(log(n))
    complexity to search and add elements.
  \begin{enumerate}
    \item the random function chooses the random intersection
    \item a for loop to iterate over the intersections and remove the
      intersections which have one or both segments in common
    \item remove the segments from the segments list with a simple erase from
      the vector class
    \item the creation of the new segments is difficult. The
      difficulty is to reorder the segments in a way that the polygon
      is still closed and does not decay in two independed polygons.
      This is guaranteed by a creation of the polygon, the same as the
      creation of the final polygon and to check if the size is equal
      of the size of the segments list.
    \item the new intersections with the new segments are calculated
      by a loop over all segments and a insert into the intersections
      map.
    \item add the segments to the segments list
  \end{enumerate}
  \item to create the final list of the segments are two loops
    necessary. the first loop builds two maps with a source -> target
    combination as key -> value, both ends of the segments were added
    as key respective value. The second map is necessary because a key
    could be occure twice but only twice and therefore two maps. The
    second loop iterates from 1 to N where N is the number of points
    in the polygon and builds the polygon by moving from source to
    target and to use the old target for the new source.
\end{enumerate}

\subsubsection{Complexity}
The implemented complexity is $\bigO(n^2*log(n))$.

\subsubsection{Parameters}
\begin{description}
  \item [--nodes] how many nodes the polygon has to have. [default: 100]
  \item [--sampling-grid] the area within the polygon could grow. [default: 1500x800]
\end{description}

\subsubsection{Examples}
\begin{figure}[ht]
  \centering

  \begin{minipage}[t]{0.4\textwidth}
    \begin{tikzpicture}[yscale=0.05,xscale=0.05]

      \setcounter{i}{1}

      \draw[->] (0,0) -- (80,0) node[below] {$x$};
      \draw[->] (0,0) -- (0,70) node[left] {$y$};

      \foreach \p in {(1,11),(33,15),(10,20),(21,10),(39,22),(48,9),(20,41)} {
        \node[point] (\arabic{i}) at \p {};
        \stepcounter{i}
      }

      \draw (1) -- (2) -- (3) -- (4) -- (5) -- (6) -- (7) -- (1);
    \end{tikzpicture}
    \caption{Random polygon}
    \label{fig:tom:base}
  \end{minipage}\hfill
  \begin{minipage}[t]{0.4\textwidth}
    \begin{tikzpicture}[yscale=0.05,xscale=0.05]

      \setcounter{i}{1}

      \draw[->] (0,0) -- (80,0) node[below] {$x$};
      \draw[->] (0,0) -- (0,70) node[left] {$y$};

      \foreach \p in {(1,11),(33,15),(10,20),(21,10),(39,22),(48,9),(20,41)} {
        \node[point] (\arabic{i}) at \p {};
        \stepcounter{i}
      }

      \draw[dotted] (1) -- (2);
      \draw[dotted] (3) -- (4);
      \draw[red] (1) -- (3);
      \draw[red] (2) -- (4);
      \draw (2) -- (3);
      \draw (4) -- (5) -- (6) -- (7) -- (1);
    \end{tikzpicture}
    \caption{First step of resolving}
    \label{fig:tom:res1}
  \end{minipage}
\end{figure}

\begin{figure}[ht]
  \centering

  \begin{minipage}[t]{0.4\textwidth}
    \begin{tikzpicture}[yscale=0.05,xscale=0.05]

      \setcounter{i}{1}

      \draw[->] (0,0) -- (80,0) node[below] {$x$};
      \draw[->] (0,0) -- (0,70) node[left] {$y$};

      \foreach \p in {(1,11),(33,15),(10,20),(21,10),(39,22),(48,9),(20,41)} {
        \node[point] (\arabic{i}) at \p {};
        \stepcounter{i}
      }

      \draw[dotted] (2) -- (3);
      \draw[dotted] (4) -- (5);
      \draw[red] (3) -- (4);
      \draw[red] (2) -- (5);
      \draw (1) -- (3);
      \draw (4) -- (2);
      \draw (5) -- (6) -- (7) -- (1);
    \end{tikzpicture}
    \caption{Second step of resolving}
    \label{fig:tom:res2}
  \end{minipage}\hfill
  \begin{minipage}[t]{0.4\textwidth}
    \begin{tikzpicture}[yscale=0.05,xscale=0.05]

      \setcounter{i}{1}

      \draw[->] (0,0) -- (80,0) node[below] {$x$};
      \draw[->] (0,0) -- (0,70) node[left] {$y$};

      \foreach \p in {(1,11),(33,15),(10,20),(21,10),(39,22),(48,9),(20,41)} {
        \node[point] (\arabic{i}) at \p {};
        \stepcounter{i}
      }

      \draw[dotted] (2) -- (5);
      \draw[dotted] (6) -- (7);
      \draw[red] (2) -- (6);
      \draw[red] (5) -- (7);
      \draw (1) -- (3) -- (4) -- (2);
      \draw (6) -- (5);
      \draw (7) -- (1);
    \end{tikzpicture}
    \caption{Third step of resolving}
    \label{fig:tom:res3}
  \end{minipage}
\end{figure}

\begin{figure}[ht]
  \centering

  \begin{minipage}[t]{0.4\textwidth}
    \begin{tikzpicture}[yscale=0.05,xscale=0.05]

      \setcounter{i}{1}

      \draw[->] (0,0) -- (80,0) node[below] {$x$};
      \draw[->] (0,0) -- (0,70) node[left] {$y$};

      \foreach \p in {(1,11),(33,15),(10,20),(21,10),(39,22),(48,9),(20,41)} {
        \node[point] (\arabic{i}) at \p {};
        \stepcounter{i}
      }

      \draw[dotted] (2) -- (5);
      \draw[dotted] (6) -- (7);
      \draw[red] (2) -- (7);
      \draw[red] (5) -- (6);
      \draw (1) -- (3) -- (4) -- (2);
      \draw (7) -- (1);
    \end{tikzpicture}
    \caption{Wrong resolving, because of two reasons. First: two segments exists before. Second: the polygon is not more complete nor simple}
    \label{fig:tom:wrong}
  \end{minipage}\hfill
  \begin{minipage}[t]{0.4\textwidth}
    \begin{tikzpicture}[yscale=0.05,xscale=0.05]

      \setcounter{i}{1}

      \draw[->] (0,0) -- (80,0) node[below] {$x$};
      \draw[->] (0,0) -- (0,70) node[left] {$y$};

      \foreach \p in {(1,11),(33,15),(10,20),(21,10),(39,22),(48,9),(20,41)} {
        \node[point] (\arabic{i}) at \p {};
        \stepcounter{i}
      }

      \draw (1) -- (3) -- (4) -- (2) -- (6) -- (5) -- (7) -- (1);
    \end{tikzpicture}
    \caption{Final simple polygon}
    \label{fig:tom:final}
  \end{minipage}
\end{figure}

\FloatBarrier
